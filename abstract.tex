%How is existing work?
%Sharing inside mulithreaading programs is not easy, they can easily cause correctness or performance problem. 
%Inappropriate sharing can dramatically degrade the performance of 
%mulithreading programs and seriously affect the scalability. 
%So detecting false sharing accurately and precisely can be helpful for user to fix corresponding performance problem. 

\begin{comment}
False sharing is notorious for performance degradation in multithreaded
programs. It apprears when two or more threads running on different cores periodically access 
different portions of data that can fit into one cache line. Since caching
system in a multicore processor needs to ensure a coherent view of memory
accross all cores, it has to grant an exclusive access
for each write operation by invidating duplicate copies in other cores. As a
result, frequent cache invalidation can seriously affect the scalability and
performance of multithreaded programs.
\end{comment} 

False sharing is a notorious performance issue for different software stacks, 
which can dramatically degrade the performance and seriously affect the scalability of 
systems.

%Many reserach efforts have been made to detect false sharing. 
Unfortunately, previous approaches to detect false sharing
either introduce significant performance overhead, or fail
to report false sharing accurately and precisely, or have different limitations of usage. 
%\sheriff{}, the prior state-of-the-art tool, 
%can only detect write-write false sharing in applications using \pthreads{} library.
We propose a novel method, \Predator{}, to combine compiler instrumentation
and runtime system to detect false sharing: 
the compiler instruments every memory access and 
the runtime system collects and analyzes memory accesses to detect false sharing problems.
\Predator{} detects false sharing accurately and precisely: it reports no 
false positives and pinpoints exact objects having false sharing problems.
Also, unlike previous work, this method can be extended to
identify false sharing problems across the entire software stack, including 
hypervisors, operating systems, libraries and applications. 
\Predator{} have successfully detected false sharing of real applications,
including \texttt{mysql} server application and \texttt{boost} library. Fixing these found
false sharing problems improves performance by $6\times$ and $40\%$ correspondingly.

Moreoever, existing tools can only detect those manifested false sharing problems.
However, occurrences of false sharing can be affected by alignments between
objects and cache lines: any change of compiler optimization, compiler, memory manager, 
memory allocation order, cache line size or different target binary 
may change alignments, and thus affect occurrences of false sharing, 
which leaves many of them undetected by existing tools.
\Predator{} is the first tool which can accurately predict possible false sharing 
with the need of occurrences. 
It can report all false sharing problems with only one execution and with reasonable overhead, 
around $7\times$ performance overhead on average.

%What is novel in our work?
%How is the performance overhead?
