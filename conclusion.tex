\label{sec:conclusion}
This paper presents a novel method to detect false sharing problem by combining
compiler instrumentation technique and a runtime system. 
Compiler instruments every read and write access of global variables
and heap objects by inserting a function call to invoke the runtime
system. 
By collecting and analyzing information passed through instrumented function calls, the runtime system 
can detect false sharing based on calculating the number of cache invalidations: those 
false sharing causing severe performance degradation has a great amount of 
cache invalidations. To differentiate false sharing with true sharing, 
\Predator{} further provides word accesses information for those cache lines involved in false sharing, 
which can help users debug and fix the problem.

\Predator{} further predicts potential false sharing caused by the change of cache line size 
or starting addresses of objects. It helps to prevent the predicament of existing tools:
problems may occur in a real environment rather than test environment due to environmental changes.
Our evaluation shows \Predator{} can effectively detect several unknown false sharing problems in two popular benchmark
suites, \texttt{Phoenix} and \texttt{PARSEC}. 
We also evaluate \Predator{} on $6$ different real applications. The results
show that it successfully detects false sharing problems inside \texttt{MySQL} and \texttt{boost} libraries. 
