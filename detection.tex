\label{sec:detection}

\subsection{Overview}
%\defaults{} is a hybrid approach that combines the runtime system with compiler instrumentation.
\label{sec:overview}
As described in Section~\ref{sec:intro}, 
false sharing only occurs when two threads running on two different cores, with its private cache,
simultaneously access independent data in the same cache line.
In the remainder of this paper, we assume that {\it each thread runs on a 
distinct core with its own private cache}. 

Given this assumption, we observe that 
{\it if a thread writes a cache line after other threads have 
accessed the same cache line, this write operation most likely causes at least
an invalidation}. 
Based on this \textbf{basic observation}, \Defaults{} captures cache invalidations for all possible 
cache lines 
in the detecting phase. Since only frequent cache invalidations can cause performance problem, 
\defaults{} ranks performance degrading severeness of false sharing according to 
the number of cache invalidations and only reports those performance-degrading false sharing.
% by keeping track of accesses from different threads. 
 
%to be loaded from main memory, wasting both the CPU time and memory bandwidth in the same time.
To capture cache invalidations, it is important to track memory accesses from different 
threads. 
Without hardware, however, it is difficult for runtime system itself to 
capture read and write accesses efficiently, accurately and timely. 
For example, \Sheriff{} relies on memory protection mechanism and word-by-word
comparison to track accumulated write accesses. But there is no way to track read accesses with  
reasonable overhead~\cite{sheriff}. 
On the other hand, compiler can easily identify  
read or write accesses. However,
it is lack of the knowledge about when and how those instructions are being accessed, 
which depend on a specific execution including input and runtime environment. 

Therefore, \defaults{} combines the runtime system with compiler instrumentation technique to capture 
cache invalidations: compiler instruments memory accesses so the runtime
system is notified when an access is executed (see Section~\ref{sec:compiler}),
and the runtime system is responsible for collecting and analyzing actual memory accesses 
to detect and report false sharing (see Section~\ref{sec:runtime}).

\subsection{Compiler Instrumentation}
\label{sec:compiler}

\Defaults{} relies on LLVM to perform instrumentation on Intermediate Representation(IR) level~\cite{llvm}.
It traverses all functions one by one and 
searches for memory accesses on global and heap variables. 
For each memory access, \defaults{} instruments one function call to 
notify the runtime system with memory access address and access type.
The reason \Defaults{} skip accesses to stack variables 
is that stack variables are normally used for be thread local storage and
therefore do not introduce false sharing. However, instrumentation on stack
variables can always be turned on when necessary. 

The instrumentation pass is placed at the very end of LLVM optimization passes 
so that only those memory accesses surviving all previous 
LLVM optimization passes are instrumented. 
This  technique is very similar to the one used in AddressSanitizer~\cite{Addresssanitizer}.

\subsection{Runtime System}
\label{sec:runtime}
The runtime system of \defaults{} collects every memory access by handling 
those functions calls inserted during compiler instrumentation phase and analyzes
possible cache invalidations based on the basic observation discussed in Section~\ref{sec:overview}.
In the end, \defaults{} precisely reports performance-degrading false sharing:
for each global variable invovled in false sharing, its name, address and size
information are reported; For heap objects, actual callsite stack for allocations, address 
and size information are reported.
In addition, \defaults{} provides word accesses information for those cache lines 
involved in false sharing, including which threads are accessing which words. 
This information can help 
users to locate where the problem is and how to fix false sharing problem.

\subsubsection{Detecting Cache Invalidations}
\Defaults{} reports those global variables or heap objects on those cache lines 
having a large amount of cache invalidations. 
Thus, it is critical to capture invalidations effectively. 
\Defaults{} achieves this goal by introducing a  
two-entries-cache-history table for each cache line, which is 
shown in Table~\ref{table:cachehistory}. 

\begin{table}
\centering
  \begin{tabular}{ l | r }
    \hline
    {Thread ID} & {Type of Access} \\ \hline
    \hline
     &   \\ \hline
     &   \\ \hline
  \end{tabular}
  \caption{Two-entries-cache-history table for every cache line. \label{table:cachehistory}}
\end{table} 

In this table, each entry has two fields: thread ID and access type (read or write). 
For every new access on a cache line $L$, we check $L$'s history table
$T$ to decide whether there is a cache invidation based on following rules:

\begin{itemize}
\item
  For a read access $R$, 
  \begin{itemize}
    \item
      If $T$ is full, there is no need to record this read access, becasue
      read accesses can not generate cache invalidations.
    \item
      If $T$ is empty, or not full and another existing entry has a different thread
      ID, then we record $R$ and its thread by filling a new entry in to the table. 
  \end{itemize}
\item
  For a write access $W$, 
  \begin{itemize}
    \item
      If $T$ is full, then $W$ can cause a cache invalidation since at least 
      one of two existing entries has a different thread ID for this cache line. 
      After recording this invalidation, we update 
      existing entry with $W$ and its thread.
    \item
      If $T$ is empty,
      then we record $W$ and its thread by filling a new entry in to the table. 
    \item
      If $T$ has an exisiting entry,
      we check if $W$ and the exisiting entry has the same thread ID. If
      so, $W$ can not cause a cache invalidation, so we update the existing
      entry with $W$. Otherwise, we identify an invalidation on this line caused by $W$. 
      After recording this invalidation information, we update 
      existing entry with $W$ and its thread.
  \end{itemize}
\end{itemize}

\subsubsection{Reporting False Sharing}
After those individual cache lines having a large amount of cache
invalidations are detected,
\defaults{} needs further analysis to differentiate actual false sharing from true sharing. 
True sharing, e.g. mulitple threads updating 
the same counter in a cache line, can also causes a large amount of cache invalidations.

In order to report false sharing precisely and accurately,  
\Defaults{} employs the following mechanisms. 
\begin{itemize}
\item
\defaults{} keeps track of access information for each word on those
cache lines involved in false sharing: 
how many reads or writes to each word by which thread. 
When a word is accessed by multiple threads,
we marked the access as a shared access. This information 
allows \defaults{} to accurately distinguish false sharing from true sharing 
in the reporting phase.
It is also helpful to locate where 
actual false sharing occurs when there are multiple fields or multiple objects 
in the same cache line, thus can greatly reduce manual overhead to fix false sharing problems
correspondingly. 

\item
In order to precisely report origins of false sharing heap objects, \defaults{}
keeps callsite information for each heap object and reports the source code level
information about each heap object. To obtain those callsite information, \defaults{}
intercepts all memory allocations and de-allocations, and relies on \texttt{backtrace()} 
function of \texttt{glibc} library to obtain the callsite stack. 
\defaults{} also avoids pseudo false sharing (false positives) caused by memory re-usages 
by handling correspondingly in memory de-allocations, whereas those heap objects with false 
sharing is not re-used at all.

\item
To keep tracking of reads and writes on each word for those problematic cache lines,
the bookkeeping information lookup must be efficient.
Similar to 
AddressSanitizer~\cite{Addresssanitizer} and other systems~\cite{qinzhaodetection}~\cite{Valgrind}. 
\defaults{} uses shadow memory mechanism 
where each cache line of user space has a corresponding entry in the shadow memory. 
Based on address of an access, \defaults{} can quickly locate
the metadata for the corresponding cache line.

\item
In order to support shadow memory mechanism, \defaults{} uses predefined starting address and 
fixed size for heap. It also contains a customized memory allocator, which is built on 
Heaplayers~\cite{heaplayers} based on a ``per-thread-heap'' mechanism firstly introduced 
by Hoard~\cite{Hoard}. Memory allocations from different threads 
are not coming from the same physical cache line, which automatically avoids false sharing 
introduced by memory allocator.
However, it also brings a shortcoming that those false sharings caused by the default memory 
allocator can not be detected by \defaults{}. 

\end{itemize} 
 
\subsection{Optimizations}
\label{optimization}
Tracking every memory access can be extermely expensive, thus 
\defaults{} also utilizes the following mechanisms to further improve performance.
%, where main improvement has been evaluated in Section~\ref{}.

\subsubsection{Threshold-Based Tracking Mechanism}
\label{sec:thresholdtracking}
\Defaults{} aims to detect performance degrading false sharing.
Since cache invalidations are the root cause of performance degrading and only writes 
can possibly introduce cache invalidations, 
those cache lines with a small amount of writes are never being our target.
For this reason, \defaults{} starts to track cache invalidations, reads and word accesses 
of a cache line only after the number of writes on this cache line is above a
pre-defined threshold, called {\it Tracking-Threshould}. 
Before this threshold is reached, \defaults{} only tracks writes on a cache line 
while skipping those reads operations. 
This mechanism helps to filter out
those impossible cache lines, reducing performance and memory overhead
in the same time.

In the actual implementation, \defaults{} maintains two mappings in shadow memory: 
{\it CacheWrites}, which tracks memory writes on every cache line, 
and {\it CacheTrackings}, which tracks detailed information 
on each cache line when the number of writes on a cache line is larger than
the {\it Tracking-Threshold}. 
If the threshold is not reached, there is no need to check corresponding {\it CacheTrackings}. 
Figure~\ref{fig:algorithm} illustrates the detailed mechanism.

\begin{figure}[!t]
\begin{lstlisting}
void HandleAccess(unsigned long addr, bool isWrite) {
 unsigned long cacheIndex=addr>>CACHELINE_SIZE_SHIFTS;
 cachetrack *track=NULL;

 if(CacheWrites[cacheIndex]<TRACKING_THRESHOLD) {
  if(isWrite) {
   if(ATOMIC_INCR(&CacheWrites[cacheIndex]) 
      ==TRACKING_THRESHOLD-1) {
    track=allocCacheTrack();
    ATOMIC_CAS(&CacheTrackings[cacheIndex],0,track));
   }
  } 
 }
 else {
  track=CacheTrackings[index]);
  if(track){
   // Track cache invalidations and detailed accesses
   track->handleAccess(addr, isWrite);
  }
 }
}
\end{lstlisting}
\caption{Pseudo-code to handle an access.\label{fig:algorithm}}
\end{figure}

To avoid expensive lock operations, \defaults{} uses atomic instruction to increment 
{\it CacheWrites} counters for each cache line. 
When the number of writes of a cache line reaches the predefined threshold,
it allocates space to track detailed cache invalidations and word accesses.
For performance reason, \defaults{} also 
uses atomic \texttt{compare-and-swap} instruction to set cache tracking address for this cache line in
the shadow mapping.
After {\it CacheWrites} on a cache line reaches predefined {\it Tracking-Threshold}, 
all read and write accesses on this cache line are started to be tracked.
Cache invalidations are also computed based on cache line history table of corresponding
cache line, shown in Table~\ref{table:cachehistory}. 


\subsubsection{Selective Compiler Instrumetation}
\defaults{} relies on instrumentation to provide memory access information to the runtime system 
and detects false sharing based on sequences of memory accesses. 
Thus the amount of instrumentation can largely affect the performance overhead: more 
instrumentation always means more performance overhead. 
Thus, \defaults{} provides a very flexible framework to instrument programs 
depending on performance requirements. 

Currently, \Defaults{} only instruments once for one type of memory access on each address 
in the same basic block. 
We argue that this selective instrumentation does not affect the effectiveness of detection. 
Because \Defaults{} targets to detect those false sharing with a large amount of cache invalidations,
less tracking of accesses inside one basic block can induce less cache invalidations 
but it won't affect the overall behavior of cache invalidations. 

% detection will not cause performance problem. 
For performance reason, 
\defaults{} can be easily extended to support more selective instrumentation as follows:
\begin{itemize}
\item
\Defaults{} can selectively instrument reads-and-writes or writes-only. Fo example, instrumenting writes-only can help to detect write-write false sharing, as Sheriff does. 
\item
\Defaults{} can be set to instrument or skip one specific kind of targets. For example, user can provide a black list so that those modules,
functions or variables are not instrumented. 
Also, user can provide a red list so that only specified functions or variables are instrumented. 
\end{itemize}

\subsubsection{Sampling Mechanism}
According to Section~\ref{sec:thresholdtracking}, 
when a cache line has writes larger than {\it Tracking-Threshold}, 
every access must be tracked to add the following bookkeeping information,
including word accesses information, access counter of 
this cache line and cache access history table, etc. 
When a cache line actually has false sharing or true sharing problems inside,
updating those counters can greatly exacerbate performance problems caused by false sharing or true
sharing: not only there is an cache invalidation of original cache line, but also there is
at least another cache invalidation caused by updating these bookkeeping information.

To reduce the performance overhead caused by this, \defaults{} only samples the first specified
number of accesses of each sampling interval for those problematic cache lines. 
Note that we originally keep a global counter for all accesses and uses the number of 
all accesses to compuate sample intervals. However, this creates 
extereme performance overhead caused by cache invalidations of updating the same counter
for all accesses. 
Currently, we maintain an access counter for each cache line 
and only tracks the first $10000$ accesses for every 1 million accesses 
(sampling interval) on a cache line.
So we uses the sampling rate 1\% for those problematic cache lines, 
with writes larger than {\it Tracking-Threshold}.
We might miss false sharing with an access pattern like this:
accesses causing false sharing must always 
happen exactly at other 99\% accesses for every 1 million accesses.
However, the possibility with this exact pattern on a cache line is extermely low.
Thus, by using this sampling mechanism, we greatly reduce the performance overhead 
while not missing any performance-degrading false sharing.  

\begin{comment}
\subsubsection{Updating-In-Place}
%During the development of \defaults{}, we discover over $2\times$ performance overhead for those benchmarks 
%without any false sharing problems. 

Originally, all memory accesses are instrumented with a library call to notify the runtime system. However, this creates
some unnecessary performance overhead from function calls and library calls. 
A library call invokes normal function call overhead plus another indirection overhead through a Global Offset Table (GOT) and 
Procedure Linkage Table (PLT).
This can introduce significant performance overhead for some simple logic listed in the following:
\begin{itemize}
\item
When total writes on a cache line is less than the pre-defined threshold, if an access is a write, 
\defaults{} simply incrementes the counter of writes for this cache line. If an access is a read,
\defaults{} does not do  anything. 
\item
When total writes on a cache line is larger than the pre-defined threshold, then this cache line is under 
tracking. As described in last section, \defaults{} only sample certain number of accesses (1\%) for every 
sample interval. Most accesses (99\%) only needs to increment an access counter of corresponding cache line.
\end{itemize}

These simple logic only invokes several circles of useful work, either checking or updating a counter. 
Thus, the overhead of function calls or library
calls can be much more than this. 
In order to avoid these overhead, \defaults{} implements these simple logic in places where these
memory accesses are instrumented. 
This brings another problem: how to choose shadow mapping addresses. 
We borrow the idea of AddressSanitizer~\cite{Addresssanitizer} by 
choosing these addresses statically. \defaults{} pre-allocated the address space between $0x40000000$ and
$0xC0000000$ for the heap. Also, \defaults{} chooses $0x200000000000$ as the starting address for the shadow mapping of cache writes 
$0x200080000000$ as the starting address for the shadow mapping of cache tracking. 
Also, \defaults{} intentionally stores the access counter in the first word of cache tracking so that 
those checking and updating can be executed in-place when an memory access is instrumented. 
\end{comment}
