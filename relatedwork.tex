\label{sec:relatedwork}

This section describes related work in false sharing detection, prevention, or both. There is no previous
system to predict unobserved false sharing.

\subsection{False Sharing Detection}
Based on the SIMICS functional simulator, Schindewolf et al.\ designed a tool 
to report different kinds of cache usage information,
such as cache miss and cache invalidations~\cite{falseshare:simulator}.
Pluto utilizes Valgrind to track the sequence of memory read and write
events on different threads and reports a worst-case estimation of
possible false sharing~\cite{falseshare:binaryinstrumentation1}.
Similarly, Liu uses Pin to collect memory access information and
reports total false sharing caused miss information~\cite{falseshare:binaryinstrumentation2}.
These tools can only report the possibility of false sharing without precise information of each
actual false sharing problem.  
Also, they impose about $100-200\times$ performance overhead.

Zhao et al.\ present a dynamic instrumentation based approach to 
detect false sharing and other cache contention problems
for multithreading programs~\cite{qinzhaodetection}. 
%This tool instruments those memory references dynamically using Umbra~\cite{Umbra}, 
%in order to track the ownership of cache lines. 
It uses a shadow memory technique to maintain memory access history and track the ownership of 
cache lines. 
However, it can only support at most $8$ threads and its memory overhead 
is above $2\times$. In addition, it cannot differentiate cold cache misses from actual false sharing problems.

Intel's performance tuning utility (PTU)~\cite{detect:ptu, detect:intel} uses Precise Event Based Sampling (PEBS) hardware support to detect false sharing problems.  It can only point out those functions that might cause false sharing, but no line number information on a false sharing problem.  PTU cannot distinguish
true sharing from false sharing. In addition, PTU aggregates memory accesses without considering the memory re-usage and access interleaving, leading to numerous false positives. Sanath et.al. designed a machine learning based approach to detect false sharing problems. They train their classifier on mini-programs and apply this classifier to general programs ~\cite{mldetect}. 
In stead of instrument memory accesses, this tool relies on hardware performance counters to collect memory accesses events. So they can achieve very low performance overhead, about 2\% performance overhead. 

In addition to their individual disadvantages,
all above approaches share a common shortcoming. 
They cannot pinpoint the exact location of false sharing in the source code, so programmers have to examine the source code and identify problems manually.

Pesterev et.al.\ describe a tool, DProf, designed to help programmers identify cache misses based on
AMD's instruction-based sampling hardware~\cite{DProf}.
DProf requires manual annotation to locate data types and object fields, and cannot detect false
sharing when multiple objects reside on the same cache line.

\subsection{False Sharing Prevention}
Jeremiassen and Eggers use a compiler transformation to automatically adjust the
memory layout of applications through padding and alignment~\cite{falseshare:compile}.
Chow et al.\ describe an approach to alter parallel loop scheduling to avoid
sharing~\cite{falseshare:schedule}.
These static analysis based approaches only works for regular,
array-based scientific codes.

Berger et al.\ describe Hoard, a scalable memory allocator that can
reduce the possibility of false sharing by making different threads
use different heaps~\cite{Hoard}. Hoard 
cannot avoid false sharing problem in global variables or within
single heap objects: the latter appear to be the primary source of
real false sharing problems.

\subsection{False Sharing Detection and Prevention}
\sheriff{}~\cite{sheriff} provides two tools to handle false sharing based on 
their ``threads-as-processes'' framework .
\Sheriff{}'s detection tool 
report false sharing accurately and precisely with only $20\%$ performance overhead.
However, it can only detect write-write type of false sharing for those programs 
using \pthreads{} library. It can also break correctness for those programs communicate across different threads using stack variables or ad hoc synchronizations (e.g. self-defined variables).
These shortcomings greatly limit their usage on real-world applications.  
\Predator{} can detect all kinds of false sharing and has no limitations on applications. \Predator{} has been utilized to find actual false sharing in real applications like MySQL and the Boost libraries.

\Sheriff{}'s prevention tool prevents false sharing from happening. Unfortunately, it only improves performance for programs with a small amount of synchronizations. 
For those programs with a large number of synchronizations, the overhead could lead to performance degradation.
%Both \SheriffDetect{} and \SheriffProtect{} assumes the correct usage of \pthreads{} APIs,  
%so they may not work correctly for those programs having ad-hoc
%synchronizations (e.g., flag synchronizations) or using stack variables to communicate among threads.

Plastic~\cite{OSdetection} leverages the sub-page granularity memory remapping facility
provided by Xen hypervisor to detect and tolerate false sharing automatically.
However, the sub-page memory remapping mechanism is not supported by most existing operating system currently, reducing its universality. In addition, Plastic cannot pinpoint the exact source of false sharing.  
In order to utilize their prevention tool, a program has be run on the Xen hypervisor, limiting the applicability of their prevention technique.
