%False sharing problem is a cache usage problem. 
%Cache, with much faster access speed than main memory, is normally utilized by CPU
%to accelerate program executions by preloading a fixed size of data into the cache each time, 
%called as a cache line.  

%%%%%%%%%%%%%%%%%%%%%%%%%%%%%%%%%%%%%%%%
% Why 
%%%%%%%%%%%%%%%%%%%%%%%%%%%%%%%%%%%%%%%%
\label{sec:intro} 

While writing correct multithreaded programs is often challenging,
making them scale can present even greater obstacles. Any source of
resource contention can impair scalability or even lead to
applications running slower as the number of threads increases.

False sharing is a particularly insidious form of resource contention.
It occurs when two threads inadvertently contend because the objects
they are accessing happen to reside on the same cache line. When at least
one of the threads frequently updates its object, the resulting cache
coherence traffic can degrade performance by up to an order of
magnitude~\cite{falseshareeffect}. Worse, false sharing is generally
difficult to detect\CC{ : SHOULD we remove this: unlike locks or truly shared objects, there is
often no obvious culprit in the source code}.

As cache lines have grown larger and multithreaded applications have
become commonplace, false sharing has become an increasingly important
problem. False sharing has been found across the software stack,
including inside the Linux kernel~\cite{OSfalsesharing}, the Java
virtual machine~\cite{JVMfalsesharing}, common
libraries~\cite{libfalsesharing} and several real
applications~\cite{mysql,appfalsesharing}.

Recent efforts to false sharing detection fall short in a number of dimensions. Some introduce excessive performance overhead, making them too slow to use in practice~\cite{falseshare:binaryinstrumentation1,falseshare:binaryinstrumentation2,falseshare:simulator}. Most do not report false sharing precisely and accurately~\cite{falseshare:binaryinstrumentation1,detect:ptu,detect:intel,falseshare:binaryinstrumentation2,DProf,qinzhaodetection}, some require special OS support~\cite{OSdetection} or impose limits on their use~\cite{sheriff}.
% or they can only detect one kind of
%false sharing problem on a specific multithreading library~\cite{sheriff}.

In addition, all of these systems share one key limitation: they can
only report \emph{observed} cases of false sharing. As Nanavati et al.\ point out, false sharing is sensitive to where objects are placed in cache lines and so can be affected by a wide range of factors~\cite{OSdetection}. For example, using the gcc compiler \emph{accidentally} eliminates false sharing in the Phoenix linear\_regression benchmark at certain optimization levels, while LLVM does not do so at any optimization level.  A slightly different memory allocation sequence (or different memory allocator) can reveal or hide
false sharing, depending on where objects end up in memory; using a different hardware platform with different addressing or cache line sizes can have the same effect. All of this means that existing tools cannot root out potentially devastating cases of false sharing that could arise with different inputs, in different execution environments, and on different hardware platforms.

%\textsc{Sheriff} avoids these limitations, but  
%can only detect write-write type false sharing and 
%and can break correctness of programs using ad-hoc synchronizations or using stack variables to 
%communicate across different threads~\cite{sheriff}. 
%Also, none of these existing approaches  has been verified to find actual 
%false sharing problems in real applications.

%Despite their different features and limitations, all existing detection tools 
%share two common drawbacks.
%Firstly, existing techniques can not be applied for 
%the entire software stack.
% although they might work on a specific level of software stack.

%To be more specific, 
%\Predator{} can report potential false sharing problems with different cache line size and different starting address of an object 
%without the need of another execution. 

This paper presents \Predator{}, a novel software-only false sharing detector that not only \emph{detects} all existing false sharing problems accurately and precisely, but also \emph{predicts} potential false sharing problems that could appear in a different execution environment.

\subsection*{Contributions}

This paper makes the following contributions:

\begin{itemize}


% prediction
\item
\textbf{Predictive False Sharing Detection:} \Predator{} is the first system that can \emph{predict} potential false sharing that does not manifest in an execution but may appear and greatly degrade the performance of programs in a slightly different environment. 
\CC{Predictive false sharing generalizes from a single
execution to identify potential false sharing instances that can arise by slight changes in object placement and alignment. }
It also can predict false sharing in hardware platforms with larger cache line sizes by tracking accesses within \emph{virtual cache lines} that span multiple physical lines. Predictive false sharing detection thus overcomes a key limitation of previous detection tools.

%Existing approaches is based on specific hardware,
%runtime environment(using specific libraries and compilers) and specific cache line size, which is OK to detect those 
%existing false sharings. But they fail to capture those variables or objects which can greately slow down performance 

%In order to save memory usage, we propose a threshold invoked detection based on the predefined number of writes on a cache line, which
%can be used to track the .

% effect: can detect all kinds of false sharing problems.
\item
\textbf{A Practical and Effective Detection Tool:} \Predator{} detects both observed and predicted false sharing 
problems accurately and precisely with reasonable overhead (average: $6\times$ performance, $2\times$ memory).  It reveals unknown false sharing problems in benchmark suites evaluated by previous approaches. It is also the first false sharing tool able to automatically and precisely uncover
false sharing problems in real applications, including 
MySQL and the Boost library. Fixing these problems 
dramatically improves performance, by $12\times$ for a benchmark and by $6\times$ for \texttt{MySQL} application.
%in the benchmarks, and $40\%$ in the actual applications).


\item
\textbf{A Fully General Approach to False Sharing Detection:} \Predator{} provides the first completely general false sharing detection method by combining compiler-based instrumentation with a runtime system. While this
paper focuses on user-level multithreaded applications, \Predator{}'s approach is broadly applicable across the entire software stack because it does not depend on specific hardware, OS, or library support.
%Since compiler can be utilized to do selective instrumentation, 
%\Predator{} can be utilized to detect all kinds of false sharing problem with ideal overhead. 

% combine compiler instrumentation with runtime system, so that it can 
%detect all kinds of false sharing problem including write-write false sharing. 
%avoid the shortcomings of runtime-only system, where Sheriff can not detect the read-writee false sharings. 
%Also, because of the limit of their implementation, Sheriff can not support the program using the ad hoc synchronization
%or using the stack variables to communicate among different threads.
%Also, it looks like that we can provide a evenly performance overhead.
%Also, sheriff should only work on specific thread library, currently, it can only work on pthreads library. 
%We are trying to extending the same idea to different thread library. Now we can also run DeFault to detect all 
%false sharings using other threads libraries.

\end{itemize}

\subsection*{Outline}

\CC{
The remainder of this paper is organized as follows. 
Section~\ref{sec:detection} describes \Predator{}'s detection mechanisms and algorithms in detail. Section~\ref{sec:prediction} discusses how to predict potential false sharing accurately without actually observing false sharing directly. Section~\ref{sec:evaluation} presents experimental results, including using \Predator{} to reveal unknown false sharing problems in several benchmarks and real applications. 
Section~\ref{sec:discussion} discusses design tradeoffs and some important facts of \Predator{}. 
Section~\ref{sec:futurework} outlines directions for future work and Section~\ref{sec:relatedwork} describes key related work. Finally, Section~\ref{sec:conclusion} concludes.}


%There are two types of false sharings:
%A. Different threads are accessed different locations according to the definition of flase sahrings. 
%B. Locations with a large amount of reads is placed in the cache line with a large amount of writes.
%For the second type, existing tools may tend to miss that.  
