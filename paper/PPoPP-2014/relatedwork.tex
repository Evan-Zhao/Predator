\label{sec:relatedwork}

This section describes related work in detecting or preventing false sharing; no prior work predicts false sharing.

\subsection{False Sharing Detection}
Schindewolf et al.\ designed a tool based on the SIMICS functional simulator to report different kinds of cache usage information, such as cache misses and cache invalidations~\cite{falseshare:simulator}. Pluto relies on the Valgrind dynamic instrumentation framework to track the sequence of memory read and write events on different threads, and reports a worst-case estimation of possible false sharing~\cite{falseshare:binaryinstrumentation1}.
Similarly, Liu uses Pin to collect memory access information, and reports total cache miss information~\cite{falseshare:binaryinstrumentation2}.
These tools impose about $100-200\times$ performance overhead.

Zhao et al.\ present a tool based on the DynamoRIO framework to detect false sharing and other cache contention problems
for multithreading programs~\cite{qinzhaodetection}. 
It uses a shadow memory technique to maintain memory access history and detects cache invalidations based on the ownership of cache lines. However, it can only support at most $8$ threads. In addition, it cannot differentiate cold cache misses from actual false sharing problems.

Intel's performance tuning utility (PTU) uses Precise Event Based Sampling (PEBS) hardware support to detect false sharing problems~\cite{detect:ptu,detect:intel}.  PTU cannot distinguish true sharing from false sharing. In addition, PTU aggregates memory accesses without considering memory reuse and access interleavings, leading to numerous false positives. Sanath et al.\ designed a machine learning based approach to detect false sharing problems. They train their classifier on mini-programs and apply this classifier to general programs~\cite{mldetect}. Instead of instrumenting memory accesses, this tool relies on hardware performance counters to collect memory accesses events. This approach operates with extremely low overhead but ties false sharing detection to a specific hardware platform.

In addition to their individual disadvantages,
all approaches discussed above share a common shortcoming:  
they cannot pinpoint the exact location of false sharing in the source code, so programmers must manually examine the source code to identify problems.

Pesterev et al.\ present DProf, a tool that help programmers identify cache misses based on AMD's instruction-based sampling hardware~\cite{DProf}. DProf requires manual annotation to locate data types and object fields, and cannot detect false sharing when multiple objects reside on the same cache line.

\subsection{False Sharing Prevention}

Jeremiassen and Eggers use a compiler transformation to automatically adjust the memory layout of applications through padding and alignment~cite{falseshare:compile}. Chow et al.\ alter parallel loop scheduling in order to avoid false
sharing~\cite{falseshare:schedule}. These approaches only works for regular, array-based scientific code.

Berger et al.\ describe Hoard, a scalable memory allocator that can reduce the possibility of false sharing by making different threads use different heaps~\cite{Hoard}. Hoard cannot avoid false sharing problem in global variables or within a single heap object: the latter appears to be the primary source of false sharing problems.

\subsection{False Sharing Detection and Prevention}

\sheriff{} provides two tools to handle false sharing based on its ``threads-as-processes'' framework~\cite{sheriff}.
\Sheriff{}'s detection tool reports false sharing accurately and precisely with only $20\%$ performance overhead.
However, it can only detect write-write false sharing, and only works for programs that use the \pthreads{} library. It can also break programs that communicate across different threads with stack variables or \emph{ad hoc} synchronizations. These shortcomings limit \Sheriff{}'s usefulness for real-world applications.  
\Predator{} can detect all kinds of false sharing and imposes no limitations on the kind of applications it works on. 

\Sheriff{}'s prevention tool prevents false sharing altogether, eliminating the need for programmer intervention. However, in programs with many synchronization calls, the overhead imposed by \Sheriff{} could lead to performance degradation.

Plastic leverages the sub-page granularity memory remapping facility provided by the Xen hypervisor to detect and tolerate false sharing automatically~\cite{OSdetection}. However, the sub-page memory remapping mechanism is not currently supported by most existing operating systems, reducing its generality. In addition, Plastic cannot pinpoint the exact source of false sharing.  
In order to utilize Plastic's prevention tool, a program has to run on the Xen hypervisor, limiting the applicability of their prevention technique.
