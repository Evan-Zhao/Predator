False sharing is a notorious problem for multithreaded applications
that can drastically degrade both performance and
scalability. Existing approaches can precisely identify
the sources of false sharing, but only report false
sharing actually observed during execution; they do not generalize
across executions. Because false sharing is extremely sensitive to
object layout, these detectors can easily miss false sharing problems
that can arise due to slight differences in memory allocation order or
object placement decisions by the compiler. In addition, they cannot
predict the impact of false sharing on hardware with different cache
line sizes.

This paper presents \Predator{}, a predictive software-based false
sharing detector. \Predator{} generalizes from a single execution to
precisely predict false sharing that is latent in the current
execution. \predator{} tracks accesses within a range that could lead
to false sharing given different object placement. It also tracks
accesses within
\emph{virtual cache lines}, contiguous memory ranges that span actual
hardware cache lines, to predict sharing on hardware platforms with
larger cache line sizes. For each, it reports the exact program
location of predicted false sharing problems, ranked by their
projected impact on performance. We evaluate \Predator{} across a
range of benchmarks and actual applications. \Predator{} identifies
problems undetectable with previous tools, including two
previously-unknown false sharing problems, with no false
positives. \Predator{} is able to immediately locate false sharing problems in MySQL and the
Boost library that had eluded detection for years.
