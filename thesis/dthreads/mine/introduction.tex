\label{sec:introduction}

The advent of multicore architectures has made multithreaded
programming increasingly necessary, but writing multithreaded programs
remains painful. It is notoriously far more challenging to write
concurrent programs than sequential ones because of the wide range of
errors it can cause, including deadlocks and race
conditions~\cite{havender,76897,130623}. Because thread interleavings
are non-deterministic, different runs of the same multithreaded
program can unexpectedly produce different results. These
``Heisenbugs'' greatly complicate debugging, and eliminating them
requires extensive testing to account for possible thread
interleavings~\cite{DBLP:conf/icse/BallBHMQ09,DBLP:conf/asplos/BurckhardtKMN10}.

% Lots of recent work on bug finding. Getting better, but still difficult.

Instead of testing, one promising alternative approach is to attack
the problem of concurrency bugs by eliminating its source:
non-determinism. A fully \emph{deterministic multithreaded system}
would prevent Heisenbugs by ensuring that executions of the same
program with the same inputs always yield the same results, even in
the face of race conditions in the code. Such a system would not only
dramatically simplify debugging of concurrent
programs~\cite{Carver:1991:RTC:624586.625040} and reduce their
attendant testing overhead, but would also enable a number of other
applications. For example, a deterministic multithreaded system would
greatly simplify record and replay for multithreaded
programs~\cite{Choi:1998:DRJ:281035.281041,LeBlanc:1987:DPP:32387.32396}
and the execution of multiple replicas of multithreaded applications
for fault
tolerance~\cite{deterministic-process-groups,1134000,224058,replicant-hotos}.

Several recent software-only proposals aim at providing
deterministic multithreading, but these all suffer from a variety of
disadvantages. Language-based approaches are effective at removing
determinism but require programmers to write code in specialized
languages, which can be
impractical~\cite{Bocchino:2009:TES:1640089.1640097,Burckhardt:2010:CPR:1869459.1869515,Simpson:1999:SEE:330346.330357}. Recent
deterministic systems that target legacy programming languages
(especially C/C++) are either incomplete or impractical. Kendo ensures
determinism of synchronization operations with low overhead, but does
not guarantee determinism in the presence of data
races~\cite{1508256}. Grace prevents all concurrency errors but is
limited to fork-join programs, and although it is efficient, it can require
code modifications to avoid large runtime
overheads~\cite{grace}. CoreDet, a compiler and runtime system,
enforces deterministic execution for arbitrary multithreaded C/C++
programs~\cite{Bergan:2010:CCR:1736020.1736029}. However, it exhibits
prohibitively high overhead (running up to $8\times$ slower
than \pthreads{}; see Section~\ref{sec:evaluation}) and generates
thread interleavings at arbitrary points
in the code, complicating program debugging and testing.

\hspace{1em} \\
\noindent
\textbf{Contributions:}
This paper presents \textbf{\dthreads{}}, an efficient deterministic runtime
system for multithreaded C/C++ applications. \dthreads{} guarantees
deterministic execution of multithreaded programs even in the presence
of data races (notwithstanding external sources of non-determinism
like I/O): given the same sequence of inputs, a program
using \dthreads{} always produces the same output. \dthreads{}'
deterministic commit protocol not only eliminates data races but also
prevents lock-based deadlocks.

\dthreads{} is easy to deploy: it works as a direct replacement for
the \pthreads{} library, requiring no code modifications or
recompilation. \dthreads{} is also efficient. Its software
architecture avoids the need for expensive write buffers, and 
its commit protocol eliminates cache-line based false sharing, a
notorious performance problem for multithreaded programs. These two
features enable \dthreads{} to nearly match or even exceed the
performance of \pthreads{} for the majority of the benchmarks examined
here. \dthreads{}
thus is a significant improvement over the state of the art in
deployability and performance, and provides evidence that fully
deterministic multithreaded programming may be practical.

% XXX borrowed from Grace, XXX borrowed from Treadmarks, etc.


%Deployable: directly replaces
%the \pthreads{} library, requiring no code modifications.

% Summary of results. Improvements over state-of-the-art (CoreDet).

The remainder of this paper is organized as
follows. Section~\ref{sec:dthreads-architecture} describes
the \dthreads{} architecture and algorithms in depth, and
Section~\ref{sec:discussion} discusses key
limitations. Section~\ref{sec:evaluation} evaluates \dthreads{}
experimentally, comparing its performance and scalability
to \pthreads{} and CoreDet. Section~\ref{sec:related-work} provides an
overview of related work, Section~\ref{sec:future-work} describes
future directions, and Section~\ref{sec:conclusion} concludes.
