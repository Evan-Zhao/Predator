\begin{figure*}[!ht]
{\centering
%\fbox{
\subfigure{\lstinputlisting[numbers=none,frame=none,boxpos=t]{fig/mainthread.example.pseudocode}}
\hspace{50pt}
\subfigure{\lstinputlisting[numbers=none,frame=none,boxpos=t]{fig/thread1.example.pseudocode}}
\hspace{50pt}
\subfigure{\lstinputlisting[numbers=none,frame=none,boxpos=t]{fig/thread2.example.pseudocode}}
%}
\caption{A simple multithreaded program with data races on \texttt{a} and \texttt{b}. With \pthreads{}, the output is non-deterministic, but \dthreads{} guarantees the same output on every execution.\label{fig:sample}}
}
\end{figure*}

\label{sec:dthreads-overview}

% \subsection{Overview}

We begin our discussion of how \dthreads{} works with an example execution of a
simple, racy multithreaded program, and explain at a high level how
\dthreads{} enforces deterministic execution.

Figure~\ref{fig:sample} shows a simple multithreaded program that,
because of data races, non-deterministically produces the outputs
``1,0,'' ``0,1'' and ``1,1.''  With \pthreads{}, the order in
which these modifications occur can change from run to run, resulting
in non-deterministic output. 

With \dthreads{}, however, this program \emph{always} produces the
same output, (``1,1''), which corresponds to exactly one possible
thread interleaving. \dthreads{} ensures determinism using the
following key approaches, illustrated in Figure~\ref{fig:architecture}:

\textbf{Isolated memory access:}
In \dthreads{}, threads are implemented using separate processes with
private and shared views of memory, an idea introduced by
Grace~\cite{grace}.  Because processes have separate address spaces,
they are a convenient mechanism to isolate memory accesses between
threads.  \dthreads{} uses this isolation to control the visibility of
updates to shared memory, so each ``thread'' operates independently
until it reaches a synchronization point (see
below). Section~\ref{sec:threadsasprocs} discusses the implementation
of this mechanism in depth.

\textbf{Deterministic memory commit:} 
Multithreaded programs often use shared memory for communication, so \dthreads{} must propagate one thread's writes to all other threads. To ensure deterministic
execution, these updates must be applied at deterministic times, and in a deterministic order.

\dthreads{} updates shared
state in sequence at synchronization points. These points
include thread creation and exit; mutex lock and unlock; condition
variable wait and signal; posix sigwait and signal; and barrier waits. Between
synchronization points, all code effectively executes within an
atomic \emph{transaction}. This combination of memory isolation between
synchronization points with a deterministic commit protocol guarantees
deterministic execution even in the presence of data races.

\textbf{Deterministic synchronization:}
\dthreads{} supports the full array of \pthreads{} synchronization
primitives.  Because current operating systems make no guarantees about the
order in which threads will acquire locks, wake from condition
variables, or pass through barriers, \dthreads{} re-implements these
primitives to guarantee a deterministic ordering.
Details on the \dthreads{} implementations of these primitives are
given in Section~\ref{sec:synchronization}.

\textbf{Twinning and diffing:}
Before committing updates, \dthreads{} first compares each modified
page to a ``twin'' (copy) of the original shared page, and then writes
only the modified bytes (diffs) into shared state (see
Section~\ref{sec:dthreads-optimizations} for optimizations that avoid
copying and diffing).  This algorithm is adapted from the distributed
shared memory systems TreadMarks and
Munin~\cite{dsm:munin,dsm:treadmarks}. The order in which threads
write their updates to shared state is enforced by a single global
token passed from thread to thread; see
Section~\ref{sec:sharedmem} for full details.

%%% \vfill %%% EDB

\subsection*{Fixing the data race example}
Returning to the example program in Figure~\ref{fig:sample}, we can
now see how \dthreads{}' memory isolation and a deterministic
commit order ensure deterministic output. \dthreads{} effectively
isolates each thread from each other until it completes, and then
orders updates by thread creation time using a deterministic
last-writer-wins protocol.

At the start of execution, thread 1 and thread 2 have the same view of
shared state, with $a = 0$ and $b = 0$.  Because changes by one thread
to the value of $a$ or $b$ will not be made visible to the other until
thread exit, both threads' checks on line 2 will be true.  Thread 1
sets the value of $a$ to 1, and thread 2 sets the value of $b$ to 1.
These threads then commit their updates to shared state and exit, with
thread 1 always committing before thread 2.  The main thread then has
an updated view of shared memory, and prints ``1, 1'' on every
execution.

This determinism not only enables record-and-replay and replicated
execution, but also effectively converts Heisenbugs into ``Bohr''
bugs, making them reproducible. In addition, \dthreads{} optionally
reports any conflicting updates due to racy writes, further
simplifying debugging.

