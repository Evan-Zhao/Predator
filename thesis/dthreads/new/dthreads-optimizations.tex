\label{sec:dthreads-optimizations}

\dthreads{} employs a number of optimizations that improve its performance.

\textbf{Lazy commit:}
\dthreads{} reduces copying overhead and the time spent in the serial
phase by \emph{lazily} committing pages. When only one thread has ever
modified a page, \dthreads{} considers that thread to be the page's
owner.  An owned page is committed to shared state only when another
thread attempts to read or write this page, or when the owning thread attempts to modify it in a later phase. \dthreads{} tracks reads with page protection and signals
the owning thread to commit pages on demand. To reduce the number of
read faults, pages holding global variables (which we expect to be
shared) and any pages in the heap that have ever had multiple writers are
all considered unowned and are not read-protected.

\textbf{Lazy twin creation and diff elimination: }
To further reduce copying and memory overhead, a twin page is only created
when a page has multiple writers during the same transaction.
In the commit phase, a single writer can directly copy its working copy to
shared state without performing a diff. \dthreads{} does this by
comparing the local version number to the global page version number
for each dirtied page.  At commit time, \dthreads{} directly copies
its working copy for each page whenever its local version number
equals its global version number.  This optimization saves the cost of a twin 
page allocation, a page copy, and a diff in the common case where just one thread is the sole writer of a page.

\textbf{Single-threaded execution: }
Whenever only one thread is running, \dthreads{} stops using memory protection 
and treats certain synchronization operations (locks and barriers) as no-ops.
In addition, when all other threads are waiting on condition variables, 
\dthreads{} does not commit local changes to the shared mapping or discard  
private dirty pages. Updates are only committed when the thread
issues a signal or broadcast call, which wakes up at least one thread
and thus requires that all updates be committed.

\textbf{Lock ownership: }
\dthreads{} uses lock ownership to avoid unnecessary waiting when threads are
using distinct locks.  Initially, all locks are unowned.  Any thread that
attempts to acquire a lock that it does not own must wait until the serial phase
to do so.  If multiple threads attempt to acquire the same lock, this lock is 
marked as shared.  If only one thread attempts to acquire the lock, this thread 
takes ownership of the lock and can acquire and release it during the parallel
phase.

Lock ownership can result in starvation if one thread continues to re-acquire an owned lock without entering the serial phase.  To avoid this, each lock has a maximum number of times it can be acquired during a parallel phase before a serial phase is required.

\textbf{Parallelization: }
\dthreads{} attempts to expose as much parallelism as possible in the
runtime system itself.  One optimization takes place at the start of trasactions, where
\dthreads{} performs a variety of cleanup tasks. These include releasing private page frames,
and resetting pages to read-only mode by calling the \texttt{madvise}
and \texttt{mprotect} system calls. If all this cleanup work is done
simultaneously for all threads in the beginning of parallel phase
(Figure~\ref{fig:phase}), this can hurt performance for some
benchmarks.

Since these operations do not affect other the behavior of other
threads, most of this work can be parallelized with other threads'
commit operations without holding the global token. With this
optimization, the token is passed to the next thread as soon as
possible, saving time in the serial phase.  Before passing the token,
any local copies of pages that have been modified by other threads
must be discarded, and the shared read-only mapping is restored.  This
ensures all threads have a complete image of this page which later
transactions may refer to.  In the actual
implementation, this cleanup occurs at the end of each transaction.

