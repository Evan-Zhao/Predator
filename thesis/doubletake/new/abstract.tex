Dynamic analysis can be helpful for debugging, but is often too
expensive to use in deployed applications. We introduce evidence-based
dynamic analysis, an approach that enables extremely lightweight
analyses for an important class of errors: those that can be forced to
leave evidence of their existence. Evidence-based dynamic analysis lets
execution proceed at full speed until the end of an epoch. It then
examines program state to find evidence that an error occurred at some
time during that epoch. If so, execution is rolled back and
re-execution proceeds with instrumentation activated to pinpoint the
error. We present \doubletake{}, a prototype evidence-based dynamic
analysis framework. We demonstrate its generality by building analyses
to find buffer overflows, dangling pointer errors, and memory
leaks. \doubletake{} is precise and efficient: its buffer overflow
analysis runs with just 2\% overhead on average, making it the fastest
such system to date.

