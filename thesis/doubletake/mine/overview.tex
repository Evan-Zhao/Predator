\label{sec:overview}

\DoubleTake{} improves performance for dynamic analysis by utilizing a record-and-replay idea, 
which is discribed in Figure~\ref{fig:overview} {\bf FIGGGURE}.. 
\DoubleTake{} divides an execution of a program into multiple epochs, at the boundaries of irrevocable
system calls (discussed in Section~\ref{sec:syscall}) or different types of exits, e.g., 
segmentation faults, aborts or normal exit.
In the beginning of an epoch, \doubletake{} takes a snapshot of program state, 
including changeable memory and registers.
During an epoch, a program proceeds at full speed with extremely light-weight recording on results of
certain system calls, in order to facilitate re-execution. 
In the end of an epoch, \doubletake{} checks the evidence for errors. 
We have implemented checks for three types of errors (discussed in Section~\ref{sec:applications}), 
including heap buffer overflows, memory usage-after-free errors, and memory leaks. 
If an error is detected, 
\doubletake{} switches to re-execution mode (described in Section~\ref{sec:re-execution}) 
to gather detailed information of the error.
If no error is detected and a program ends this epoch because of an irrevocable system call, 
\doubletake{} will issue the irrevocable system call and start a new epoch.
Because \doubletake{} only check memory errors in the end of epochs and ends epochs rarely, 
which amortizes the cost of checking errors over 
long periods of uninterrupted execution, 
\doubletake{} achieves extremely light performance overhead.

Checking evidence of errors and gathering detailed error information are application specific. 
In order to detect buffer overflows and memory usage-after-free errors, \doubletake{} fills 
memory with canaries, which is discussed in Section~\ref{sec:canaries}.
In order to precisely identifying the instruction reponsible for these two errors, 
\DoubleTake{} installs hardware watch points (see Section~\ref{sec:watchpoints}) before 
actual re-executions. 
In order to reproduce those found memory errors deterministically, 
it is crucial for \doubletake{} to have exactly the same memory allocations in re-executions. 
Because it is difficult to do this in a general memory allocator,
\DoubleTake{} designs its custom memory allocator for all memory allocations and deallocations, 
which is discussed in Section~\ref{sec:allocator}. 

\subsection{System Calls}
\label{sec:syscall}

% Why we need to care about system calls. Because some system calls are irrevocable.
% However, not all IO are irrevocable. For example, if a file reads twice, as long as 
% it returns the same result. We do not need to undone the results of IO operations. 
\doubletake{} ends epochs on irrevocable system calls: the execution before
and after irrecocable system calls belong to different epochs.
Irrevocable here has a different meaning as that in transactional memory programming. 
In transactional memory programming, irrevocable actions include IO and system calls whose
effect cannot be rollbacked or completely invoked in user space~\cite{Irrevocabletrans}.
\doubletake{} relaxes the definition of irrevocable: only the results of system calls cannot be 
reproduced are considering to be irrevocable system calls. 
System calls are classified to the following types.

\begin{table}[t]
	\centering
	\small
	\renewcommand{\arraystretch}{1.5}
	\begin{tabular}{r|p{6cm}}
		\textbf{Category} & \textbf{Functions} \\
		\hline
		
		Repeatable		& \texttt{getpid}, \texttt{sleep}, \texttt{pause}\\
		
		Recordable		& \texttt{mmap}, \texttt{gettimeofday}, \texttt{time}, 
						  \texttt{clone} , \texttt{open}\\
		
		Checkpointed		& \texttt{write}, \texttt{read} \\
		
		Delayed			& \texttt{close}, \texttt{munmap} \\
		
		Irrevocable		& \texttt{fork}, \texttt{exec}, \texttt{exit}, \texttt{lseek}, \texttt{pipe}, \texttt{flock}, \texttt{socket related system calls}\\
	\end{tabular}
	\caption{
		System calls handled by \doubletake{}. Other calls not listed are treated as irrevocable, and will end the current epoch. Section~\ref{sec:syscalls} describes how \doubletake{} handles calls in each category.
		\label{tbl:syscalls}
	}
\end{table}

\label{sec:syscalls}
\paragraph{Repeatable system calls} do not modify system state, and will always return the same result.
This category includes \texttt{sleep}, \texttt{pause}, and system calls to query system's state, e.g. \texttt{getpid()}, \texttt{fstat} and \texttt{stat}. 
\doubletake{} does not need any special handling for these system calls.
	
\paragraph{Recordable system calls} may return different results if they are re-executed. 
\doubletake{} records the result of these system calls. 
During re-execution, \doubletake{} will simply return the saved result 
instead of re-executing the system call. 
This category includes \texttt{mmap()}, \texttt{gettimeofday()}, \texttt{time()}, and \texttt{clone()}.

\paragraph{Checkpointable system calls} modify system state, 
but \doubletake{} can save this state beforehand and restore it prior to re-execution. 
Most of file IO related system calls fall into this category. 
For example, the \texttt{write()} system call can modify the contents of a file and moves the 
file positions inside operating system.
They can be treated as irrevocable system calls, but that can greatly affect performance 
because of intensive usages in applications.
\doubletake{} saves those file positions of openning files in the beginning of each epoch and 
recovers those file positions before re-execution. 
%Because \texttt{read()} and \texttt{write()} can be used in socket communications, which is considered
%as irrevocable system calls, \doubletake{} checks whether those file descriptors are related to 
%actual files. If they are invoked on actual files, there is no other operation on those file related system calls.  
	
\paragraph{Delayable system calls} will irrevocably change program state, but can safely 
be delayed until the end of the current epoch. 
\doubletake{} delays all calls to \texttt{munmap()} and \texttt{close()}.
	
\paragraph{Irrevocable system calls} cannot be replayed. \doubletake{} must end the current epoch 
before these system calls are allowed to proceed. A system call not belonging to previous categories
can be conservatively treated as an irrevocable system call.

\vspace*{\baselineskip} 
During normal execution of an epoch, \doubletake{} intercepts all calls to \pthreads{} 
library functions that may issue system calls. \doubletake{} actually handles much more function calls
than the listed number of system calls. For example, \texttt{open()}, \texttt{fopen()}, \texttt{fdopen}, \texttt{freopen()}, and \texttt{creat()} can possibly call \texttt{open} system calls. 
Then all of these library functions should be intercepted. 
%We assume that other libraries are built 
%on this library: there is no direct system call from other libraries or applications.

To reduce overhead, we should reduce the amount of epochs as much as possible by 
recording, checkpointing and delaying system calls since starting and ending an epoch 
involves large performance overhead, as discussed in Section~\ref{sec:implementation}, 
Currently, \doubletake{} only optimizes those system calls invoked by evaluated applications.
Current category of system calls can be seen in Table~\ref{tbl:syscalls}. Other system calls
not mentioned here belong to the category of irrevocable system calls. 

\begin{comment}
To support multithreading programs, \DoubleTake{} has to record the order of synchronizations 
and replay them in the same order (Section~\ref{sec:multithreading}) and design memory allocator specially (Section~\ref{sec:mtheap}).

The basic idea of \DoubleTake{} is to greatly {\em reduce the number of checkings}:  
instead of checking a possible out-of-bounds error before every memory access, 
\DoubleTake{} checks all possible out-of-bounds errors of the whole program 
before those irrevocable system calls or in the end of an execution.
By checking buffer overflows accumulatively, \DoubleTake{} amortizes 
checking overhead over long executions and achieves much better performance. 
As showed in Figure~\ref{fig:diagram}, if a program does not have buffer overflows, 
it continues to perform those system calls and 
starts a new epoch after them by snapshotting state of this execution. 
If a program is found to have buffer overflows, \DoubleTake{} re-executes it by rolling back 
to last good state after installing hardware watchpoints on those overflowing addresses.
By handling exceptions, 
\DoubleTake{} can pinpoint precisely those memory accesses causing memory overflows without
actually instrumenting every access. 
\end{comment}


\subsection{Canary Mechanism}
\label{sec:canaries}
Canaries, also called as sentinels, are normally put before or after each actual object.
They are initialized to a special value in the beginning so that modifications of those values
indicates the problems.
Canaries have been used by many previous works to
detect buffer overflows ~\cite{overflow:purify, exterminator, overflow:lbc}.
Similarly, \stopgap{} puts canaries before and after heap objects, which can be used .
to detect buffer underflows and overflows.
The size of canary is a balance between memory overhead and precision.
As discusssed by Hasabnis et al. ~\cite{overflow:lbc}, larger size helps to detect more
buffer overflows, but increasing memory overhead.
\stopgap{} chooses the size of a pointer as the size of each word-based canary:
4 bytes for 32 bit binaries and 8 bytes for 64 bit binaries.
For those objects with size not aligned by words, \stopgap{} filles byte-based canaries before a
word-based canary.

\stopgap{} maintains a global bitmap to mark placements of those canaries:
if a word includes canaries, either a word-based canary or multiple
byte-based canaries, its corresponding bit is set to $1$.
Since \stopgap{} only uses a bit for each word, this helps to reduce memory overhead of the bitmap.
Previous works use a bit for each byte in order to capture one-byte buffer overflow~\cite{overflow:lbc, AddressSanitizer}.
However, it is uncessary to do this since \stopgap{} embeds size information of each heap object
into the header, thus by coming with size information and byte-based canaries \stopgap{} doesn't
loose the precision of detecting one-byte buffer overflows.
Reducing the size of bitmap helps improving the speed of heap integrity checking too.

\subsection{Watch Point Mechanism}
\label{sec:watchpoints}
Watch point mechanism relies on hardware debug registers to monitor memory accesses on
specific addresses. Previous work has used this mechanism for their specific 
targets \cite{fastboundschecking, Kivati}.

Debug registers are universally supported by most if not all existing CPUs, such as
X86 and X86-64 architecture, PowerPC, MIPS, ARM, etc. 
The main target of debug registers is to support software debuggers, e.g. gdb. 
In X86 and X86-64 architectures, there are four debug address registers, one debug 
status register and one debug control register. 
A user can set up debug address registers to hold those addresses that want to monitor 
before executions of a program. 
Then this user can be notified by an exception whenever a memory access matches one of those 
debugging address in debug registers. 

%\DoubleTake{} installs the addresses of canaries with buffer overflows and watches memory
%accesses on them in the re-execution phase by handling those exceptions generated by debug registers. 
%By analyzing the context of signal handler, 
%\DoubleTake{} can precisely locate those instructions causing buffer overflows and report to users.
%More implementation details are discussed in Section~\ref{sec:installwatchpoints}. 

\subsection{Customized Heap Allocator}
\label{sec:allocator}
A general memory allocator invokes a big number of \texttt{mmap} or \texttt{sbrk} system calls,
thus it is hard to repeat memory usages of a program without recording all memory allocations. 
%related system calls and repeating
%them in re-execution phase.
\DoubleTake{} utilizes customized memory allocator for its heap allocations to overcome these shortcomings. The memory allocator of \DoubleTake{} is built on Heap Layers ~\cite{heaplayers}, 
%repeat memory allocations in re-execution phase.

%However, it is normally impossible to do this for multithreading programs.
\DoubleTake{} pre-allocates a fixed size of memory 
from its underlying operating system using \texttt{mmap} system calls and 
satisfies memory allocations from this block of memory.
All heap objects has the size of {\it power of $2$}, known as block size. 
When an object is allocated, \doubletake{} adds a object header to each object, including block size,
actual object size and canary.
When an object is deallocated, it is put into a corresponding free list of the memory allocator, 
holding objects with the same block size. 
\DoubleTake{} never changes size of an object so there is no split and merge operation on heap objects.
If size of an allocation is less than {\it power of 2}, 
\DoubleTake{} allocates an object with the size of next {\it power of 2}.
% but putting canaries
%right after this actual object in order to detect .
%The support for multithreading programs can be seen in Section~\ref{sec:mtheap}.

% why it is deterministic?
Since it is conventient to find out memory usage conditions in the beginning of each epoch 
and all memory allocation are deterministic according to the order of program execution from the same heap without involving other system calls,  
it is relatively easy to reproduce memory usage in production runs.
% Because of this design, it is convenient to take a snapshot in production runs 
%and repeat memory usage in reproduction runs. 


