\subsection{Limitations}
\doubletake{} can detect those heap buffer overflows overwritting ``guard zones'', 
either caused by direct memory access or caused by
incorrect library calls (e.g. \texttt{memcpy} or \texttt{strcpy}). 
\doubletake{} can detect most of heap underflows: they can be detected if they are
aligned by \texttt{power of 2} or they are freed in the end of program. 
For those objects that they are not freed and not aligned, \doubletake{} can not detect them.
 
\doubletake{} can not detect buffer overflows on the globals and the stack 
since \doubletake{} can not change the layout of globals by inserting guard zones around them.
AddressSanitizer is a complementary approach for this: we can rely on AddressSanitizer 
to put guard zones for globals, then \doubletake{} can check the overflow of the globals accumulatively. 

\doubletake{} can not detect the usage-after-free memory error, which already implemented by 
AddressSanitizer. But their mechanism can be implemented in our framework easily and provide
this detection probability. 

\textbf{We need multiple runs for programs with race conditions}.
Although the idea of using re-execution and watch point can be applied to multithreaded programs,
\doubletake{} has not targeted for multithreaded programs currently. 

\doubletake{} can not detect those non-continguous buffer overflows. 
\subsection{Future Work} 
The first possible work of \doubletake{} is to extend it on multithreaded programs because
multithreaded programs is universal these days. 
However, there are some challenges to do this: how we can repeat the execution of 
a multithreaded program given random memory accesses from different threads;
how to handle those synchronizations without violating the shared-memory
semantics. 

Another possible work is to extend \doubletake{} to find more memory errors, such as dangling pointer 
errors, memory usage-after-free errors.
We possibly have to change the memory allocator in order to achieve memory usage-after-free errors
since we are utilizing the first words of freed memory blocks in the free list management. 

The last but not least, we can combine the \doubletake{} mechanism and \texttt{gdb} mechanism
to provide an interactive debugging system without actually using \texttt{gdb}. It can invoke this 
interactive debugging mechanism whenever bugs are found. 
