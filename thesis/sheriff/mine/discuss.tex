\label{sec:discussion}

\subsection{Performance}
\label{discussion-perf}
The performance of one tool is very important for people when they decide to use that.
Sheriff tried the best to improve the performance.
%, now the overhead 
%of Sheriff comparing to use \textit{pthreads} libarary on original program is only 26\%. 

Sheriff takes the following ways to improve the performance. 
\begin{enumerate}
\item
Sheriff only use the protected heap to meet those allocations smaller than 
cores number multiply the cache line size. 
Protections on all heap area are very expensive. 
We believe that small objects has a greater
possibility to be the candidates of false sharing problems. 
Our experiments from \texttt{phoenix} and \texttt{PARSEC} suites also confirms our guess. 

\item 
Sheriff remove protection when there is just one thread in system. If there is one thread, 
it is impossible to cause false sharing problems so it didn't make sense to protect those 
memory. Sheriff only starts the protection when there is a \texttt{pthread\_create()} call 
to start new thread. Also, Sheriff closes the protection
timely when it detects the uniqueness of one thread in case of \texttt{pthread\_join()}.

\item
Sheriff uses the sampling mechanism to capture continuous writes in the same transaction. 
In the timer handler, Sheriff should compare those written pages(dirty) to check the writes
on them. Since false sharing problems can happen on those pages which are shared by multiple process
simultaneously, Sheriff only check those shared pages only to reduce the checking overhead. 
Thus Sheriff designed a shared page-based array to save the status of 
all pages. When one page is faulted because of a write operation, Sheriff increments the written-users
on this page.
Thus we can much time to do the page checking and improve the performance greately.


\item 
Sheriff use the shared page version number to improve performance on commits. 
Notice that Sheriff are trying to compare corresponding ``twin'' page and ``working'' page 
in order to get modifications by one transaction and 
commits corresponding modifications to the ``shared'' mapping.
Actually, if one private page's corresponding shared mapping hasn't been modified by other threads, 
it is safe to commit all content of this page to shared mapping. It is faster to copy 
the whole private page's content to the shared mapping of this page. 
In order to detect whether there is some conflicts, Sheriff introduce the version number for those pages in shared mapping.
Sheriff increments page's version number (initially version number is 0) when there is a commit on 
one page. 

Before the creation of ``twin'' page, Sheriff saves corresponding page version number for one page. 
In the end of transaction, if the saved page version number is still the same as the shared page version number, that means
no one else have committed a new version to corresponding shared mapping, it is safe to use current ``working'' page 
as next version of shared mapping. 
In this way, there is no need to use word-by-word copy but use the whole page copy instead.
\end{enumerate}

\subsection{Limitation}
According to Section~\ref{simulation:sharememory}, Sheriff doesn't try to share the stack between different threads. 
Theoritically, different threads can use stack to communicate between different threads since different threads are 
sharing the same address space. But in fact, different threads has its own stack and 
it doesn't make sense to use the private stack to communicate between different threads. 

\subsection{Precision}
Sheriff won't introduce any false positive which is totally different from the previous tools. Sheriff can differentiate 
true sharing and false sharing problem (see Section~\ref{detection:avoidfalsepositive}) and avoid false positives caused by heap objects (Section~\ref{detection:object}). 

So another questions is that whether Sheriff won't have any false negatives, especially 
those performance-related false sharing problems. 
Sheriff won't report those false sharing with low number of interleaving writes, since it won't affect the performance.
False negatives can occured in the following cases:

\begin{itemize}
\item
Sheriff is based on a Hoard-liked memory allocator,
instead of using pthreads memory allocator. That means, any false sharing problem introduced by 
pthreads memory allocator can be undetected by Sheriff. 
We argue that we are not trying to detect the false sharing problems caused by memory allocator, that is not our target.
%Besides, using some Hoard-like memory allocator can be very helpful to improve the performance of multi-threaded programs
%and that should be the standard for memory allocator for multi-threaded programs.
 
\item
According to previous section,
Sheriff choose to protect small objects only in order to improve performance. 
If one false sharing problems unfortunately happens on those large objects, then Sheriff cannot detect 
the false sharing problems. Although our evaluating benchmarks doesn't show up more false sharing objects 
when we are protecting all objects, some manual synthetic false sharing problems can intentionally do that.  

\item
Since we choose to use sampling mechanism to capture those continuous writes from different threads, it is possible
that Sheriff can miss those writes in the middle of sampling intervals. 
But we argue that Sheriff is trying to detect
those performance related false sharing problem, 
it is very unlikely that one performance related false sharing problem only
has frequent writes in the middle of sampling intervals. 
\end{itemize}

As a conclusion, we believe that Sheriff can capture most of those performance related false sharing problems. 
