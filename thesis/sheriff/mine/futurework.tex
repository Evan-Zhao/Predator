\label{futurework}
Sheriff has been implemented successfully to find some false sharing bugs in actual multithreaded applications.
The next thing of Sheriff is trying to extend current framework to find more performance related problems. 
For example, if one frequently read word happens to be in the same cache line with one frequently written words. 
Then it is better to separate these two words. But in current framework, Sheriff can not detect the memory read 
operation using twin page mechanism.
Then it will be helpful if Sheriff is combining with some debug register mechanism provided by hardware. When one
word or one object is detected to have numerous writes, then we 
can install the watchpoints for the conjacent words (in the same cache line). 
Then it is helpful to another kind of problems that we can't find in current Sheriff framework.
Also, using the watchpoints can be useful to capture those program counter to touch specific address, then
Sheriff can also present the programmer with some proof of false sharing problems. 

Another direction of Sheriff is using current framework to detect the race problems. Since we already talked in
Section~\ref{falseshare:memorywrites}, Sheriff can be used to capture writes of different threads. 
Also, now Sheriff can simulate the running of multithreaded program using process.
It is possible for Sheriff to check the race conditions if one thread is found to modify the same placement 
modified by other threads. 

Another direction to improve the performance of Sheriff is to integrate the 
hardware support of cache line fault 
mechanism if there are support of this type of hardware. 
Then we can avoid to protect the whole page. Thus 
it is also unnecessary to compare the whole pages like current mechanism, 
which is expected to improve the performance of Sheriff greately.

\begin{comment}

In order to futher improve the performance, there are two ways to do that. First, we intend to use page-based version number 
to identify whether one page is modified or not. If one page is not modified by other threads at all, there is no need to do
the compare and copy operation at all. We can do this by checking the version number. If the version number is the same as 
the shared version, that means other threads do not change this page, this page should be consistent (no need to do consistent 
checking on this page). Also, it is convenient to just change the shared mapping of this page to current page. It is safe to do 
so since current thread is holding the global lock to commit. 
Second, it will be good to improve the performance if we can identify those pages that passing from one thread to another thread.
We can track those using information according to callsites since callsite information are always the same even given with 
different inputs. 
One way is to allocate those pages, which can be used only by one process at a time, from a separate area. Then we don't need
pay overhead to those pages which can be accessed only by one process at a time. If those pages number are huge, it is
expected to improve the performance greately by doing this way.


Another direction is to extend the usage of Sheriff in the more and more popular architecture - NUMA architecture, because false sharing problem of NUMA architecture can cause more significant performance problems comparing to CMP architecture when
false sharing occurs in different cores (which have different memory), then the invalidation of the whole cache line occurs amongdifferent cores through the internal bus.

It is better to extend Sheriff if Sheriff can correct the program's race related errors automatically. Then there is no need of
\end{comment}
