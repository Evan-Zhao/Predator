\label{sec:intro}
\begin{comment}
The false sharing problems can affect the performance greately, detection of false sharing is very important? \\
Definition of false sharing\\
How is state of art? Some approaches try to avoid that(compilation, scheduling, heap allocator), but none of them 
can avoid all false sharing problems successfully. \\
That is why we need the detection tool. What is the shortcomings of current detection tool, performance, false positives, far from the problems \\
What is the contribution of Sheriff? one runtime system to simulate multi-threaded program, low overhead, no false positives, precisely locate the root of problem\\
What is the target of Sheriff? How many types of false sharing Sheriff can detect? \\ 
What is the organization of this paper? \\
%Another good part of this paper is that we can avoid the problem caused by heap objects as much as possible. For example, we 
% can try to allocate the object in the same place and we can cleanup the data for heap objects. 
\end{comment}

Multithreading is considered as one way to take advantage of those 
computation resources provided by multicore processors by spawning multiple threads 
within the context of a single process~\cite{multithread}. 
Unfortunately, writing efficient multithreaded programs are still a challenging task. 
False sharing are one of major source of performance problem. 
False sharing occurs when different threads (on different cores) are accessing on
different words in the same cache line. 
Cache coherence protocol forces one cache line to invalidate if some part of this cache line
has been modified by another processor. False sharing can force one core to wait for 
unnecessary updates from another processor, thus waste the CPU time and precious memory bandwidth. 
False sharing is a well-known performance issue on multi-core machine with 
separate caches~\cite{falseshare:Analysis, falseshare:effect}. 
One microbenchmark shows that false sharing can degrade the performance 
down to 160X slower (see Fig.~\ref{fig:benchmark2}).

\begin{comment}
There are some approaches to avoid the false sharing effects by
adjusting memory layout using compiler~\cite{falseshare:compile}, 
selecting different schedule parameters for parellel loop~\cite{falseshare:schedule}, 
and providing new memory allocator~\cite{BergerMcKinleyBlumofeWilson:ASPLOS2000}.  
Unfornately, these methods cannot avoid false sharing problems completely. 
\end{comment}
In order to detect the false sharing problems,
there are some approaches
by using simulator~\cite{falseshare:simulator}, 
binary instrumentation technique~\cite{falseshare:binaryinstrumentation1, falseshare:binaryinstrumentation2}, 
or relying on performance monitor unit hardware~\cite{detect:ptu, detect:intel}.
But those approaches can suffer from numerous false postives 
or significant performance overhead (about 200X slower). 

Besides that, all previous approaches has the same problems. 
First, they have false positives (see Fig.~\ref{fig:benchmark5} for an example) 
on dynamic heap objects(one major source of false sharing problems).
Since they don't intercept memory allocation or de-allocation operations, those tools will un-correctly 
aggregate objects access information when one address is re-used by multiple objects.
Second, all of previous tools fail to pinpoint the source of false sharing problems, 
at best pointing to particular addresses accessed by functions,
which still need a lot of manual effort to find where and why false sharing occurs.

Sheriff is designed to overcome all previous shortcomings metioned above. 
Sheriff is a software-only system (relying on hardware's page fault mechanism only) 
which doesn't rely on advanced hardware support to find the problem.
Sheriff is based on a runtime system which replaces the pthread library and can eleminating 
false sharing silently.
Sheriff won't introduce any false positives, thus programmer don't need any
unnecessary work to confirm that problem. 
Sheriff can precisely locate false sharing problem, 
by providing name and size information for global variables and providing allocation sites for heap objects, 
thus it is easy to fix the false sharing problem by these information (see one case study in 
Section~\ref{evaluation:comparison}). 
Besides, the performance overhead of Sheriff is only 26\% (???) on average, which is notably lower than
previous tools using binary instrumentation 
technique~\cite{falseshare:binaryinstrumentation1, falseshare:binaryinstrumentation2}.

% contribution
% multi-process. Which is appear firstly in Grace. Allow .
% flase sharing. Using the copy and write. Show some results. 
% Race condition.
% Some good .

Here is our main contributions in this paper:
\begin{itemize}
\item The first contribution of Sheriff is its design of a runtime system which replaces 
the pthread library and can eliminate false sharing problem silently if no synchronization.
Sheriff tries to replace threads with processes, thus the process's 
complete separation and memory protection mechanism can be combined with
``twin'' page mechanism and sampling mechanism to capture interleaving writes from different threads.
 
\item
To the best of our knowledge, Sheriff is the first tool which can detect performance-related false sharing problems 
without any false positives.
By intercepting all memory allocation and de-allocation operations, Sheriff can cleanup  
invalid access counting caused by memory re-usage for heap objects thus prevent false positives caused by incorrectly
aggregated invalid counters.
By recording corresponding threads ID on every write operation of those shared cache line, Sheriff can 
guarantee to differentiate true sharing problems with false sharing problems. 
By recording interleaving writes only, Sheriff can focus on those performance-related false sharing problems only. 

\item 
Sheriff is the first detection tool which can pinpoint exact root of false sharing problems of multithread programs
without any help of special hardware and any modification of applications, 
which can be used in general OS with general hardware.
By pinpointing exact object, Sheriff can save much manual overhead to figure out false sharing problems. 

%\item 
%Sheriff introduce the ``twin page'' mechanism which is used widely in distributed share memory 
%system~\cite{dsm:treadmarks}, and use the difference to detect the memory writes. 
%As far as we know, this mechanism is firstly used to detect the false sharing problem in multi-threaded problem. 

\end{itemize}
The remainder of this paper is organized as follows. Section~\ref{sec:overview} gives an overview
of Sheriff, including assumptions, observations, targets and two mechanisms. 
Section~\ref{sec:simulation} describes how to replace threads with process 
while maintain the same sementics of multithreaded program, which is the basics of Sheriff.
Section~\ref{sec:falseshare} explains how to detect the false sharing problems using this runtime system. 
After that, Section~\ref{sec:discussion} discusses some performance improvement methods on Sheriff, 
some limitation to use Sheriff and discusses the precision gurantee of Sheriff.
Section~\ref{sec:evaluation} then presents experimental results on some microbenchmarks 
and two well-known multithreaded benchmark suite (\textbf{phoenix} and \textbf{PARSEC}), 
then also shows false sharing found by Sheriff in these suites.
Section~\ref{sec:relatedwork} shows some related work, Section~\ref{sec:future} describes future directions. 
