\label{sec:relatedwork}

The proliferation of multicore systems has increased interest in tool
support to detect false sharing, since standard profilers like
OProfile~\cite{oprofile} or gprof~\cite{gprof} only report overall
cache misses.

\paragraph{Simulation and Instrumentation Approaches:} Schindewolf describes a system based on the SIMICS functional
simulator that reports full cache miss information, including
invalidations caused by
sharing~\cite{falseshare:simulator}. Pluto
uses the Valgrind framework to track the sequence of load and store
events on different threads and reports a worst-case estimate of
possible false sharing~\cite{falseshare:binaryinstrumentation1}. Similarly,
Liu uses Pin to collect
memory access information and then reports total false sharing miss
information~\cite{falseshare:binaryinstrumentation2}.

Independently and in parallel with this work, Zhao et al.\ developed a
tool designed to detect false sharing and other sources of cache
contention in multithreaded applications~\cite{zhao:vee:2011}. This
tool uses shadow memory to track ownership of cache lines and cache
access patterns. It is currently limited to at most 8 simultaneous
threads. Like Liu above, it only reports an overall false sharing rate
for the whole program, placing the burden on programmers to examine
the entire source base to locate any instances of false
sharing.

Unlike \sheriffdetect{}, these systems generally suffer from high
performance overhead ($5-200\times$ slower) or memory overheads.
They cannot differentiate true sharing
from false sharing, yielding numerous false positives. Because they
operate at the binary level, they can all be misled by aliasing due to
memory object reuse. Finally, and most importantly, they do not point
to the objects responsible for false sharing, limiting their value to
the programmer.

\paragraph{Sampling-Based Approaches:}
Intel's performance tuning utility (PTU)~\cite{detect:ptu,
detect:intel} uses event-based sampling, allowing it to operate
efficiently. PTU can discover shared physical cache lines, and can
identify possible false sharing at the grain of individual function
calls. PTU suffers from a high false positive rate caused by aliasing
due to reuse of heap objects, and reports false sharing instances that
have no impact on performance. PTU cannot differentiate true from
false sharing or pinpoint the source of false sharing problems,
unlike \sheriffdetect{}. Section~\ref{sec:evaluation} contains an
extensive empirical comparison of PTU to \sheriffdetect{}
demonstrating PTU's relative shortcomings.

Pesterev et al.\ describe DProf, a tool that leverages AMD's
instruction-based sampling hardware to help programmers identify the
sources of cache misses~\cite{DProf}. DProf requires manual annotation
to locate data types and object fields, and cannot detect false
sharing when multiple objects reside on the same cache line. By
contrast, \sheriffdetect{} is architecture independent, requires no
manual intervention, and precisely identifies false sharing regardless
of the flow of data or which data types are involved.

\paragraph{False Sharing Avoidance: }
In some restricted cases, it is possible to eliminate false sharing,
obviating the need for detection. Jeremiassen and Eggers describe a
compiler transformation that adjusts the memory layout of applications
through padding and alignment~\cite{falseshare:compile}.  Chow et al.\
describe an approach that alters parallel loop scheduling to avoid
sharing~\cite{falseshare:schedule}. The effectiveness of these
static analysis based approaches is primarily limited to regular,
array-based scientific codes, while \sheriffprotect{} can prevent false
sharing in any application.

Berger et al.\ describe Hoard, a scalable memory allocator can reduce
the likelihood of false sharing of distinct heap
objects~\cite{BergerMcKinleyBlumofeWilson:ASPLOS2000}. Hoard limits
accidental false sharing of entire heap objects by making it unlikely
that two threads will use the same cache lines to satisfy memory
requests, but this has no effect on false sharing within individual heap
objects, which \sheriffprotect{} avoids.

\paragraph{Other Related Work:}
\sheriff{} borrows the
process-as-thread model pioneered by Grace~\cite{grace}, but otherwise
differs from it in its semantics, generality, and goals.

Grace is a process-based approach designed to prevent concurrency
errors, such as deadlock, race conditions, and atomicity errors by
imposing a sequential semantics on speculatively-executed
threads. Grace supports only fork-join programs without inter-thread
communication (e.g., condition variables or barriers), and rolls back
threads when their effects would violate sequential semantics.

\sheriff{} extends Grace to handle
arbitrary multithreaded programs; for example, Grace is incompatible
with any applications that employ inter-thread communication,
including the PARSEC benchmarks we examine here. \sheriff{} applies
diffs at synchronization points in the program to enable updates
without rollback, giving it far greater performance (but different
semantics) than Grace. \sheriff{} does not eliminate concurrency
errors, but instead allows applications to selectively track updates
and isolate memory, enabling tools like \sheriffdetect{}
and \sheriffprotect{}.
