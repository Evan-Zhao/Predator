\documentclass[10pt]{article}

\usepackage{times}
\usepackage{epsfig}
\usepackage{graphicx}
\usepackage{subfigure}
\usepackage[usenames]{color}
\usepackage{hyperref}
\usepackage{enumitem}
%\usepackage{ifpdf}
\usepackage{amssymb}
\usepackage{inputenc}
%\usepackage{fullpage}
\usepackage[left=1in,top=1in,right=1in,bottom=1in,nohead]{geometry}

%\addtolength{\oddsidemargin}{-.5cm}
%\addtolength{\evensidemargin}{-.5cm}
%\addtolength{\textheight}{1.0cm}
%\setlength{\textheight}{8.5in}
%\setlength{\textwidth}{6in}

\definecolor{darkblue}{rgb}{0.0,0.3,0.7}
\definecolor{black}{rgb}{0,0,0}

\hypersetup{
  colorlinks,
  letterpaper=true,
  linkcolor=darkblue,
  citecolor=darkblue,
  urlcolor=darkblue,
%  linkcolor=black,
%  citecolor=black,
%  urlcolor=black,
  breaklinks=true,
%  bookmarks=false,
  pdfpagelayout=SinglePage,
  pdfstartview=XYZ 0 1000 1,
%  pdfpagemode=None,
%  pdfpagelayout=OneColumn\definecolor{darkblue}{rgb}{0.0,0.3,0.7}
}

\hyphenation{Arch-i-pel-ago}

\renewcommand{\labelitemi}{\raisebox{0.01cm}{$\blacktriangleright$}}
\newcommand{\pointer}{\raisebox{0.01cm}{$\blacktriangleright$} }

\begin{document}

  \centerline{DEFAULTS User Manual}
  \centerline{\small \sc Tongping Liu}
%  \centerline{\small \today}

  \setlength{\parindent}{1.5em}
  \vspace{0.7cm}

\section*{}

\section{Compiling the llvm compiler}
%\begin{enumerate}[label=\Alph*]
\begin{enumerate}
\item
mkdir llvm-clang-src

\item
cd llvm-clang-src

\item
wget http://llvm.org/releases/3.2/llvm-3.2.src.tar.gz

\item
wget http://llvm.org/releases/3.2/clang-3.2.src.tar.gz

\item
tar zxvf llvm-3.2.src.tar.gz 

\item
tar zxvf clang-3.2.src.tar.gz 

\item
mv clang-3.2.src llvm-3.2.src/tools/clang

\item
cd llvm-3.2.src

\item
patch -Np1 $<$ ../../falsesharing/project/compiler/instrumenter.llvm-3.2.patch 

%#vi build.sh ==> change prefix and libdir to target directory (e.g., ~/git/llvm-3.2-build)
%#./build.sh

\item
mkdir -p llvm-3.2-build

\item
cd llvm-3.2-build

\item
../llvm-clang-src/llvm-3.2.src/configure -{}-prefix=/home/tianc/git/llvm-3.2-build/\
            -{}-sysconfdir=/etc          \
            -{}-libdir=/home/tianc/git/llvm-3.2-build/lib/llvm     \
            -{}-enable-optimized         \
            -{}-enable-shared            \
            -{}-enable-targets=all       \
            -{}-disable-assertions       \
            -{}-disable-debug-runtime    \
            -{}-disable-expensive-checks 
\item
make -j8 \& make install

\end{enumerate}

\section{Compiling the runtime system (library)}
\begin{enumerate}
\item
cd runtime
\item
make
\end{enumerate}

\section{Compiling a program}
Here, we are using the patched clang to compile a program. 
The following step is using memtest for an example. 
\begin{enumerate}
\item 
cd memtest
\item
 \$(CLANG\_DIR)/bin/clang -Wl,\$(RUNTIME\_DIR)/runtime/libdefault64.so -finstrumenter -g thread\_memtest.c -lpthread -o thread\_memtest.

In this step, we should make sure the following things. \$(CLANG\_DIR) and \$(RUNTIME\_DIR) should be replaced by real directory name.
\begin{itemize}
\item
First, we must use the patched clang to compile the program (specified in \$(CLANG\_DIR));

\item
Second, we should specify explicitely the runtime system 
by ``-Wl,\$(RUNTIME\_DIR)/runtime/libdefault64.so''.

\item
Third, we must specify the flag ``-finstrumenter''.
\end{itemize}

\end{enumerate}

\section{Compiling and Running PARSEC benchmarks}
We assume to use new making system, provided under ``evaluation'' package.
The original way to run PARSEC can be seen in http://parsec.cs.princeton.edu/.

In this package, there are 3 different directories, a script and some basic rules of makefile(Defult.mk). 
\begin{itemize}
\item
datasets: it is used to hold different datasets for different benchmarks.
\item
tests: source code of different benchmarks.
\item
include: some header files used by some Phoenix benchmarks.

\item
run\_benchmarks.py: script to run a benchmark or all benchmarks. 

\item
Default.mk: basic rules of makefile.
Normally, we should make sure the directory of runtime system. 
CXX\_DEFAULT = clang -Wl,\$(RUNTIME\_DIR)/libdefault64.so -finstrumenter
CC\_DEFAULT = clang -Wl,\$(RUNTIME\_DIR)/libdefault64.so -finstrumenter -std=gnu89


\end{itemize}

In order to run PARSEC benchmarks, we first have to set corresponding datasets in the beginning. 

The parsec sourcecode and its datasets can be got from http://parsec.cs.princeton.edu/download/2.1/parsec-2.1.tar.gz.

Here is an example to copy datasets. We are using the ``dedup'' as an example here.  

\begin{enumerate}
\item
tar zxvf parsec-2.1.tar.gz

\item
Go to \$(EVALUATION)/datasets.

\item
mkdir dedup

\item
cp parsec-2.1/pkgs/kernels/dedup/inputs/input\_native.tar ./ \& tar xvf input\_native.tar


\item 
Go to \$(EVALUATION)/tests/dedup.
After we do this, we make sure that the file or directory after decompressed has the same name as that showed in TEST\_ARG of dedup Makefile.

For example, Makefile of dedup showed liked this:
"TEST\_ARGS = -c -p -f -t $(THREADS) -i $(DATASET\_HOME)/dedup/media.dat -o output.dat.ddp".

\item
To run the false sharing detection tools on dedup benchmark, we can run ``make eval-defaults''. 
To run normal pthreads program, we can run ``make eval-pthread``.

\end{enumerate}


%\begin{thebibliography}{}
%\end{thebibliography}

\end{document}

